\documentclass[twocolumn]{article}

\usepackage[top=1.4cm,bottom=1cm,left=1.3cm,right=1cm]{geometry}
\usepackage{fontspec}
\usepackage{listings}
\usepackage{fancyhdr}
\usepackage{amsthm}
\usepackage{amssymb}
\usepackage{changepage}
\usepackage{color}
\usepackage{amsmath}
\usepackage{amssymb}
\usepackage{amsthm}

\lstset{
language=C++,
basicstyle=\ttfamily,
numberstyle=\footnotesize,
stepnumber=1,
numbersep=5pt,
backgroundcolor=\color{white},
showspaces=false,
showstringspaces=false,
showtabs=false,
frame=false,
tabsize=2,
captionpos=b,
breaklines=true,
breakatwhitespace=false,
escapeinside={\%*}{*)},
morekeywords={*},
literate={\ \ }{{\ }}1
}

\begin{document}

\setlength\parindent{0pt}

\tableofcontents

\pagestyle{fancy}
\fancyfoot{}
\fancyhead[L]{WiwiHo}
\fancyhead[C]{Star Burst Stream!!!!!}
\fancyhead[R]{\thepage}

\section{Template}
\lstinputlisting{code/header.cpp}

\section{Data Structure}

\subsection{Binary Indexed Tree}
\lstinputlisting{code/binary-indexed-tree.cpp}

\subsection{Disjoint Set Union-Find}
\lstinputlisting{code/dsu.cpp}

\subsection{Segment Tree}
\lstinputlisting{code/segment-tree.cpp}

\section{Graph}

\subsection{Dijkstra}
\lstinputlisting{code/dijkstra.cpp}

\subsection{Floyd-Warshall}
\lstinputlisting{code/floyd-warshall.cpp}

\subsection{Kruskal}
\lstinputlisting{code/kruskal.cpp}

\subsection{Tarjan SCC}
\lstinputlisting{code/scc.cpp}

\subsection{SPFA}
\lstinputlisting{code/spfa.cpp}

\section{String}

\subsection{KMP}
\lstinputlisting{code/kmp.cpp}

\subsection{Z Value}
\lstinputlisting{code/z-value.cpp}

\section{Geometry}

\subsection{Vector Operations}
\lstinputlisting{code/vector.cpp}

\subsection{Convex Hull}
\lstinputlisting{code/convex-hull.cpp}

\section{Number Theory}

\subsection{Prime Sieve}
\lstinputlisting{code/prime-sieve.cpp}

\section{DP Trick}

\subsection{Dynamic Convex Hull}
\lstinputlisting{code/dynamic-convex-hull.cpp}

\section{Numbers and Math}

\subsection{Fibonacci}

$$f(n)=f(n-1)+f(n-2)$$

\begin{equation*}
    \begin{bmatrix}
        f(n) \\
        f(n - 1)
    \end{bmatrix}
    =
    \begin{bmatrix}
        1 & 1 \\
        1 & 0
    \end{bmatrix}
    \begin{bmatrix}
        1 \\
        0
    \end{bmatrix}
\end{equation*}

\begin{tabular}{r|lllll}
    1 & 1 & 1 & 2 & 3 & 5 \\
    6 & 8 & 13 & 21 & 34 & 55\\
    11 & 89 & 144 & 233 & 377 & 610 \\
    16 & 987 & 1597 & 2584 & 4181 & 6765\\
    21 & 10946 & 17711 & 28657 & 46368 & 75025 \\
    26 & 121393 & 196418 & 317811 & 514229 & 832040 \\
    31 & 1346269 & 2178309 & 3524578 & 5702887 & 9227465 \\
\end{tabular}

$f(45) \approx 10^9$\\
$f(88) \approx 10^{18}$

\subsection{Catalan}

$$C_0=1, C_n=\sum_{i=0}^{n-1} C_i C_{n-1-i}$$
$$C_n=C_n^{2n}-C_{n-1}^{2n}$$

\begin{tabular}{r|lllll}
    0 & 1 & 1 & 2 & 5 & 14 \\
    5 & 42 & 132 & 429 & 1430 & 4862 \\
    10 & 16796 & 58786 & 208012 & 742900 & 2674440 \\
    15 & 9694845 & 35357670 & 129644790 & 477638700 & 1767263190
\end{tabular}

\subsection{Geometry}

\begin{itemize}
    \item Heron's formula:\\ The area of a triangle whose lengths of sides is $a$,$b$,$c$ and $s = (a + b + c) / 2$ is $\sqrt{s(s-a)(s-b)(s-c)}$.
    \item Vector cross product:\\ $v_1 \times v_2 = |v_1||v_2| \sin \theta = (x_1 \times y_2) - (x_2 \times y_1)$.
    \item Vector dot product:\\ $v_1 \cdot v_2 = |v_1||v_2| \cos \theta = (x_1 \times y_1) + (x_2 \times y_2)$.
\end{itemize}

\subsection{Prime Numbers}

First 50 prime numbers:\\
\begin{tabular}{r|llllllllll}
    1 & 2 & 3 & 5 & 7 & 11 & 13 & 17 & 19 & 23 & 29\\
    11 & 31 & 37 & 41 & 43 & 47 & 53 & 59 & 61 & 67 & 71\\
    21 & 73 & 79 & 83 & 89 & 97 & 101 & 103 & 107 & 109 & 113\\
    31 & 127 & 131 & 137 & 139 & 149 & 151 & 157 & 163 & 167 & 173\\
    41 & 179 & 181 & 191 & 193 & 197 & 199 & 211 & 223 & 227 & 229
\end{tabular}

Very large prime numbers:\\
\begin{tabular}{ccccc}
    1000001333 & 1000500889 & 2000000659 & 900004151 & 850001359
\end{tabular}

\subsection{Number Theory}

\begin{itemize}
    \item Inversion:\\ $aa^{-1} \equiv 1 \pmod{m}$. $a^{-1}$ exists iff $\gcd(a,m)=1$.
    \item Linear inversion:\\ $a^{-1} \equiv (m - \lfloor\frac{m}{a}\rfloor) \times (m \bmod a)^{-1} \pmod{m}$
    \item Fermat's little theorem:\\ $a^p \equiv a \pmod{p}$ if $p$ is prime.
    \item Euler function:\\ $\phi(n)=n \prod_{p|n} \frac{p-1}{p}$
    \item Euler theorem:\\ $a^{\phi(n)} \equiv 1 \pmod{n}$ if $\gcd(a,n) = 1$.
    \item Extended Euclidean algorithm:\\
    $ax+by=\gcd(a,b)=\gcd(b, a \bmod b)=\gcd(b, a-\lfloor\frac{a}{b}\rfloor b)=bx_1+(a-\lfloor\frac{a}{b}\rfloor b)y_1=ay_1+b(x_1-\lfloor\frac{a}{b}\rfloor y_1)$
    \item Divisor function:\\ $\sigma_x(n) = \sum_{d|n}d^x$. $n=\prod_{i=1}^r p_i^{a_i}$.\\ $\sigma_x(n)=\prod_{i=1}^r \frac{p_i^{(a_i+1)x}-1}{p_i^x-1}$ if $x \neq 0$. $\sigma_0(n)=\prod_{i=1}^r (a_i+1)$.
\end{itemize}

\end{document}